\documentclass[uplatex,dvipdfmx,11pt,a4paper]{jsarticle}

% --- 基本パッケージ ---
\usepackage[uplatex,deluxe]{otf} 
\usepackage[noalphabet]{pxchfon} 
\usepackage{amsmath, amssymb}
\usepackage[fleqn,tbtags]{mathtools} 
\usepackage{url}
\usepackage{float} 
\usepackage[margin=25mm]{geometry}
\usepackage{booktabs}
\usepackage{tabularx}
\usepackage{enumitem}
\usepackage{ascmac}
\usepackage{xurl}
\usepackage[dvipdfmx]{graphicx}
\usepackage[dvipdfmx]{hyperref}
\usepackage{pxjahyper} 

\hypersetup{
    hidelinks % 枠線と色をすべて非表示にする
}

% --- 文書情報 ---
\title{webプログラミングレポート課題}
\author{25G1051 近藤巧望}
\date{\today}

\begin{document}

\maketitle
\tableofcontents
\newpage

\section{はじめに}
本システムは,ゲーム「原神」における「星5キャラクター所持数一覧」「星4・5武器所持数一覧」「スコア40以上の聖遺物一覧」を
それぞれ管理するためのシステムである.各項目の新規登録,詳細表示,編集,および削除機能を提供する.
特に,聖遺物一覧のシステムに関しては,聖遺物セットごとに部位を管理し,セット効果の発動状況を考慮した上で,
聖遺物スコアが40以上のものを一覧表示する機能を備えている.そのため,本仕様書では,聖遺物セットの管理機能と所持聖遺物の管理機能を分けて記載する.
githubリポジトリは以下のURLにて公開している.
\begin{itemize}
    \item \url{https://github.com/kondou33/webpro_06}
\end{itemize}

\part{利用者用仕様書}
\setcounter{section}{0}
\section{システム概要}
本システムは,ゲーム「原神」における以下の3つの要素を管理するためのものである.
\begin{itemize}
    \item 星5キャラクター所持数一覧
    \item 星4・5武器所持数一覧
    \item スコア40以上の聖遺物一覧
\end{itemize}
各要素に対して,新規登録,詳細表示,編集,および削除の機能を提供する.
聖遺物一覧に関しては,聖遺物セットごとに部位を管理し,セット効果の発動状況を考慮した上で,
聖遺物スコアが40以上のものを一覧表示する機能を備えている.

\section{目的}
本システムの目的は,ユーザーが自身の所持する星5キャラクター,星4・5武器,および聖遺物を効率的に管理できるようにすることである.
星5キャラクターの管理は,キャラクターの育成状況や役割を把握するために役立ち,
星4・5武器の管理は,装備の最適化や戦略の立案に寄与する.
武器はレアリティが高いものほど強力ではあるが,キャラクターとの相性も重要であることや,
星4武器でも優秀なものが存在することから,星4・5武器の両方を管理できるようにしている.
聖遺物の管理に関しては,セット効果の発動状況を考慮することで,より効果的なビルド構築を支援する.
これらの管理機能を通じて,ユーザーはゲーム内での戦力強化や戦略的なプレイを促進できることを目的としている.

\section{主な機能}
本システムは,以下の主要な機能を提供する.
\begin{itemize}
    \item 星5キャラクター所持数一覧の管理
    \begin{itemize}
        \item 新規登録
        \item 詳細表示
        \item 編集
        \item 削除
    \end{itemize}
    \item 星4・5武器所持数一覧の管理
    \begin{itemize}
        \item 新規登録
        \item 詳細表示
        \item 編集
        \item 削除
    \end{itemize}
    \item スコア40以上の聖遺物一覧の管理
    \begin{itemize}
        \item 聖遺物セットの管理
        \begin{itemize}
            \item セット名称の新規登録
            \item セット名称の削除
        \end{itemize}
        \item 所持聖遺物の管理
        \begin{itemize}
            \item 新規登録
            \item 詳細表示
            \item 編集
            \item 削除
        \end{itemize}
    \end{itemize}
\end{itemize}

\section{利用方法}
\subsection{星5キャラクター一覧表示}
まずは,次の図\ref{fig:chara_list}のように,google Chromeなどのウェブブラウザで
\url{http://localhost:8080/chara}にアクセスすることで,星5キャラクター一覧表示画面が表示される.
\begin{figure}[H]
    \centering
    \includegraphics[width=0.8\textwidth]{chara.png}
    \caption{Chromeでの検索例}
    \label{fig:chara_list}
\end{figure}
アクセスすると,図\ref{fig:chara}のような一覧表示の画面が表示される.
画面下部の「追加」ボタンをクリックすると,新規登録画面へ遷移する.
また,各キャラクター名のリンクをクリックすると,詳細表示画面へ遷移する.
詳細画面では,編集ボタンをクリックすると編集画面へ遷移し,削除ボタンをクリックすると削除処理が実行される.
\begin{figure}[H]
    \centering
    \includegraphics[width=0.8\textwidth]{chara_list.png}
    \caption{星5キャラクター一覧表示画面}
    \label{fig:chara}
\end{figure}

\subsection{星4・5武器一覧表示}
星4・5武器一覧表示画面へのアクセス方法および操作方法は,星5キャラクター一覧表示画面と同様である.
\url{http://localhost:8080/weapon}にアクセスすると,図\ref{fig:weapon}のような一覧表示の画面が表示される.
\begin{figure}[H]
    \centering
    \includegraphics[width=0.8\textwidth]{weapon_list.png}
    \caption{星4・5武器一覧表示画面}
    \label{fig:weapon}
\end{figure}
使用方法も同様であり,画面下部の「追加」ボタンをクリックすると新規登録画面へ遷移し,
各武器名のリンクをクリックすると詳細表示画面へ遷移する.
詳細画面では,編集ボタンをクリックすると編集画面へ遷移し,削除ボタンをクリックすると削除処理が実行される.

\subsection{聖遺物一覧表示}
聖遺物一覧表示画面へのアクセス方法も,星5キャラクター一覧表示画面と同様である.
\url{http://localhost:8080/artifact}にアクセスすると,図\ref{fig:artifact}のような一覧表示の画面が表示される.
\begin{figure}[H]
    \centering
    \includegraphics[width=0.8\textwidth]{artifact1.png}
    \caption{聖遺物一覧表示画面}
    \label{fig:artifact}
\end{figure}
図\ref{fig:artifact}の聖遺物一覧表示は,聖遺物セットごとに部位を管理しているため,星5キャラクター一覧表示画面や
星4・5武器一覧表示画面とは構造が異なり,操作方法も異なる.
セット一覧ページの画面下部にある「追加」ボタンをクリックすると,聖遺物セット名称の新規登録画面へ遷移する.
各セット名のリンクをクリックすると,そのセットに所属する聖遺物部位一覧画面へ遷移する(図\ref{fig:artifact2}).
\begin{figure}[H]
    \centering
    \includegraphics[width=0.8\textwidth]{artifact2.png}
    \caption{聖遺物部位一覧画面}
    \label{fig:artifact2}
\end{figure}
部位一覧画面の画面下部にある「追加」ボタンをクリックすると,聖遺物部位の新規登録画面へ遷移する.
各部位名のリンクをクリックすると,詳細表示画面へ遷移する.
詳細画面では,編集ボタンをクリックすると編集画面へ遷移し,削除ボタンをクリックすると削除処理が実行される.

\part{管理者用仕様書}
\setcounter{section}{0}
\section{システムの概要と目的}
本システムは,Node.jsおよびExpressフレームワークを用いた,原神のキャラクター・武器・聖遺物データの管理ツールである.
管理者がローカル環境またはサーバー上で本プログラムを実行することで,ユーザーに対しWebブラウザ経由でのデータ登録・閲覧・編集・削除(CRUD機能)を提供する.
主な目的は,複雑な聖遺物のステータスおよび会心スコアを一元管理し,育成の効率化を図ることにある.

\section{システムの流れ}
各システムのページ遷移は以下の通りである.
\begin{figure}[H]
    \centering
    \includegraphics[clip,viewport=30 550 580 780,width=0.9\textwidth]{chara.pdf}
    \caption{星5キャラクター一覧表示のページ遷移図}
    \label{fig:page_transition}
\end{figure}
\begin{figure}[H]
    \centering
    \includegraphics[width=0.9\textwidth]{weapon.pdf}
    \caption{星4・5武器一覧表示のページ遷移図}
    \label{fig:page_transition_weapon}
\end{figure}
\newpage
\begin{figure}[H]
    \centering
    \includegraphics[width=0.9\textwidth]{artifact.pdf}
    \caption{聖遺物一覧表示のページ遷移図}
    \label{fig:page_transition_artifact}
\end{figure}
ページ遷移について,各システム共通の仕様として,最初の一覧表示画面から各詳細画面へ遷移し,
詳細画面から編集画面へ遷移する形と,一覧表示から新規登録画面へ遷移する形を採用している.
星5キャラクター一覧表示および星4・5武器一覧表示では,まず一覧表示画面から詳細画面へ遷移し,
詳細画面から編集画面へ遷移する形を採用している.
一方,聖遺物一覧表示では,聖遺物セットごとに部位を管理しているため,
一覧表示画面からセットの部位一覧画面へ遷移し,部位一覧画面から詳細画面へ遷移する形を採用している.

\section{運用方法}
本システムを運用するには,Node.jsおよびExpressフレームワークがインストールされた環境が必要である.
更に,冒頭で添付したGitHubリポジトリから\texttt{webpro\_06}フォルダをクローンまたはダウンロードし,
ターミナルで\texttt{webpro\_06}フォルダに移動した後,以下のコマンドを実行して必要なパッケージをインストールする.
\begin{verbatim}
    npm install
\end{verbatim}
依存パッケージのインストールが完了したら,以下のコマンドを実行してアプリケーションを起動する.
\begin{verbatim}
    node app5.js
\end{verbatim}
アプリケーションが起動したら,Webブラウザで\url{http://localhost:8080/chara}にアクセスすることで,
星5キャラクター一覧表示画面が表示される.
同様に,星4・5武器一覧表示画面は\url{http://localhost:8080/weapon},
聖遺物一覧表示画面は\url{http://localhost:8080/artifact}にアクセスすることで表示される.

\section{エラー処理}
システムが予期しない動作をした際や,不適切な操作が行われた場合は以下の挙動をとる. 
\begin{itemize}
    \item \textbf{存在しないデータの参照}: 削除済みのデータや存在しないIDにアクセスしようとした場合,システムは自動的に「一覧画面」へとリダイレクトを行う.これにより,画面が真っ白になったりシステムが停止したりすることを防いでいる.
    \item \textbf{入力不備の防止}:登録フォームでは,必須項目の入力漏れや数値以外の入力に対して制限を設けている.不適切な状態で送信ボタンを押した場合,ブラウザ側で警告が表示され,正しく入力されるまで登録処理は行われない.
\end{itemize}

\part{開発者用仕様書}
\setcounter{section}{0}
\section{システム概要}
本システムは,Node.jsおよびExpressフレームワークを用いた,原神のキャラクター・武器・聖遺物データの管理ツールである.
管理者がローカル環境またはサーバー上で本プログラムを実行することで,ユーザーに対しWebブラウザ経由でのデータ登録・閲覧・編集・削除(CRUD機能)を提供する.
主な目的は,複雑な聖遺物のステータスおよび会心スコアを一元管理し,育成の効率化を図ることにある.

\section{目的と機能}
本システムの目的は,ユーザーが自身の所持する星5キャラクター,星4・5武器,および聖遺物を効率的に管理できるようにすることである.
星5キャラクターの管理は,キャラクターの育成状況や役割を把握するために役立ち,
星4・5武器の管理は,装備の最適化や戦略の立案に寄与する.
武器はレアリティが高いものほど強力ではあるが,キャラクターとの相性も重要であることや,
星4武器でも優秀なものが存在することから,星4・5武器の両方を管理できるようにしている.
聖遺物の管理に関しては,セット効果の発動状況を考慮することで,より効果的なビルド構築を支援する.
これらの管理機能を通じて,ユーザーはゲーム内での戦力強化や戦略的なプレイを促進できることを目的としている.

\section{HTTPメソッドとリソース名一覧}

\subsection{星5キャラクター所持数一覧}
\begin{tabularx}{\textwidth}{llX}
\toprule
メソッド & リソース名 & 機能概要 \\
\midrule
GET & \textsf{/chara} & 星5キャラクター一覧表示 \\
GET & \textsf{/chara/create} & 新規登録フォーム表示 \\
POST & \textsf{/chara\_add} & 新規追加処理 \\
GET & \textsf{/chara/:id} & 詳細表示 \\
GET & \textsf{/chara/edit/:id} & 編集フォーム表示 \\
POST & \textsf{/chara\_update/:id} & 編集処理 \\
POST & \textsf{/chara\_delete/:id} & 削除処理 \\
\bottomrule
\end{tabularx}

\subsection{星4・5武器所持数一覧}
\begin{tabularx}{\textwidth}{llX}
\toprule
メソッド & リソース名 & 機能概要 \\
\midrule
GET & \textsf{/weapon} & 星4・5武器一覧表示 \\
GET & \textsf{/weapon/create} & 新規登録フォーム表示 \\
POST & \textsf{/weapon\_add} & 新規追加処理 \\
GET & \textsf{/weapon/:id} & 詳細表示 \\
GET & \textsf{/weapon/edit/:id} & 編集フォーム表示 \\
POST & \textsf{/weapon\_update/:id} & 編集処理 \\
POST & \textsf{/weapon\_delete/:id} & 削除処理 \\
\bottomrule
\end{tabularx}

\subsection{スコア40以上の聖遺物一覧}

\subsubsection{聖遺物セットの管理}
\begin{tabularx}{\textwidth}{llX}
\toprule
メソッド & リソース名 & 機能概要 \\
\midrule
GET & \textsf{/artifact} & 聖遺物セット一覧表示 \\
GET & \textsf{/artifact/set\_create} & セット名称入力画面表示(静的HTMLへの遷移) \\
POST & \textsf{/artifact/set\_add} & セット追加処理 \\
POST & \textsf{/artifact/set\_delete/:set\_id} & セット削除処理 \\
\bottomrule
\end{tabularx}

\subsubsection{所持聖遺物の管理}
\begin{tabularx}{\textwidth}{llX}
\toprule
メソッド & リソース名 & 機能概要 \\
\midrule
GET & \textsf{/artifact/set/:set\_id} & 指定したセットの部位一覧表示 \\
GET & \textsf{/artifact/item\_create/:set\_id} & 部位追加画面表示(IDのバケツリレー) \\
POST & \textsf{/artifact/item\_add/:set\_id} & 部位追加処理 \\
GET & \textsf{/artifact/item/:id} & 部位詳細表示 \\
GET & \textsf{/artifact/item/edit/:id} & 部位編集画面表示 \\
POST & \textsf{/artifact/item\_update/:id} & 部位更新処理 \\
POST & \textsf{/artifact/item\_delete/:id} & 部位削除処理 \\
\bottomrule
\end{tabularx}

\section{データ構造}
本システムは「管理番号」および「名称」をキーとして管理を行う.詳細画面では以下のデータ項目を保持する.

\subsection{星5キャラクターデータ}
\begin{itemize}[noitemsep]
    \item \textsf{chara\_id} (数値): 管理番号
    \item \textsf{chara\_name} (文字列): キャラ名
    \item \textsf{element} (文字列): 元素
    \item \textsf{limited\_chara} (文字列): 限定/恒常の区分
    \item \textsf{levelbonus\_name} (文字列): 突破ボーナス項目名
    \item \textsf{levelbonus\_value} (数値): 突破ボーナス値
    \item \textsf{weapon\_type} (文字列): 武器種
    \item \textsf{level} (数値): キャラレベル
    \item \textsf{nomal\_talent\_level} (数値): 通常攻撃天賦レベル
    \item \textsf{skill\_talent\_level} (数値): 元素スキル天賦レベル
    \item \textsf{burst\_talent\_level} (数値): 元素爆発天賦レベル
    \item \textsf{role} (文字列): 役割(アタッカー/サポーター等)
    \item \textsf{totu} (数値): 命の星座凸数
\end{itemize}

\subsection{星4・5武器データ}
\begin{itemize}[noitemsep]
    \item \textsf{weapon\_id} (数値): 管理番号
    \item \texttt{weapon\_name} (文字列): 武器名
    \item \textsf{weapon\_type} (文字列): 武器種
    \item \textsf{limited} (文字列): 限定/恒常の区分
    \item \textsf{rank\_level} (数値): レアリティ(4または5)
    \item \textsf{level} (数値): レベル
    \item \textsf{base\_attack} (数値): 基礎攻撃力
    \item \textsf{sub\_stat} (文字列): サブステータス項目名
    \item \textsf{sub\_stat\_value} (数値): サブステータス値
    \item \textsf{passive\_name} (文字列): 武器効果説明文
    \item \textsf{refinement\_level} (数値): 精錬ランク(1--5)
    \item \textsf{motif} (文字列): モチーフ対象キャラ名
\end{itemize}

\subsection{聖遺物データ}
聖遺物については,聖遺物セットデータと所持聖遺物データの2つに分けて管理する.
\subsubsection{聖遺物セットデータ}
\begin{itemize}[noitemsep] 
    \item \textsf{set\_id} (数値): セットID(紐付け用) 
    \item \textsf{set\_name} (文字列): セット名(例:黄金の劇団) 
    \item \textsf{set2\_effect} (文字列): 2セット効果説明 
    \item \textsf{set4\_effect} (文字列): 4セット効果説明 
\end{itemize}

\subsubsection{所持聖遺物データ}
\begin{itemize}[noitemsep]
    \item \textsf{artifact\_id} (数値): 管理番号 
    \item \textsf{set\_id} (数値): 所属するセットのID 
    \item \textsf{slot} (文字列): 部位(花・羽・砂時計・杯・冠) 
    \item \textsf{main\_stat} (文字列): メインステータス項目 
    \item \textsf{main\_stat\_value} (数値): メインステータス値 
    \item \textsf{sub\_stat1} (文字列): サブステータス1項目 
    \item \textsf{sub\_stat1\_value} (数値): サブステータス1値 
    \item \textsf{sub\_stat2} (文字列): サブステータス2項目 
    \item \textsf{sub\_stat2\_value} (数値): サブステータス2値 
    \item \textsf{sub\_stat3} (文字列): サブステータス3項目 
    \item \textsf{sub\_stat3\_value} (数値): サブステータス3値 
    \item \textsf{sub\_stat4} (文字列): サブステータス4項目 
    \item \textsf{sub\_stat4\_value} (数値): サブステータス4値 
    \item \textsf{score} (数値): 聖遺物スコア.計算式は式(\ref{eq:artifact_score})参照. 
\end{itemize}

聖遺物スコアとは,キャラクターの火力向上に寄与するステータスを数値化したものであり,以下の式で算出される.
聖遺物を使用するキャラクターのビルドに応じて,重視するステータスの重み付けが異なるため,
攻撃力,元素チャージ効率,HP,防御力などのステータスは考慮せず,
本システムでは会心率と会心ダメージに着目し,会心スコアとして定義している.
\begin{equation}
    \text{Score} = (\text{会心率} \times 2) + (\text{会心ダメージ} \times 1)
    \label{eq:artifact_score}
\end{equation}

\section{ファイルごとの役割}
本システムは,ロジックを担う \textsf{app5.js} と,各リソースの描画を担う \textsf{EJS} テンプレートファイル群および\textsf{HTML} ファイル群で構成されている.
後述の表にて,各ファイルの役割を示す.
\subsection{星5キャラクター一覧表示システム}
\begin{table}[H]
    \centering
    \begin{tabular}{ll}
    \toprule
    ファイル名 & 役割 \\
    \midrule
    app5.js & メインロジック(ルーティング,CRUD処理) \\
    chara.ejs & 星5キャラクター一覧表示画面テンプレート \\
    chara\_new.html & 星5キャラクター新規登録画面テンプレート \\
    chara\_detail.ejs & 星5キャラクター詳細表示画面テンプレート \\
    chara\_edit.ejs & 星5キャラクター編集画面テンプレート \\
    \bottomrule
    \end{tabular}
    \caption{星5キャラクター一覧表示システムのファイル構成}
\end{table}
\subsection{星4・5武器一覧表示システム}
\begin{table}[H]
    \centering
    \begin{tabular}{ll}
    \toprule
    ファイル名 & 役割 \\
    \midrule
    app5.js & メインロジック(ルーティング,CRUD処理) \\
    weapon.ejs & 星4・5武器一覧表示画面テンプレート \\
    weapon\_new.html & 星4・5武器新規登録画面テンプレート \\
    weapon\_detail.ejs & 星4・5武器詳細表示画面テンプレート \\
    weapon\_edit.ejs & 星4・5武器編集画面テンプレート \\
    \bottomrule
    \end{tabular}
    \caption{星4・5武器一覧表示システムのファイル構成}
\end{table}
\subsection{聖遺物一覧表示システム}
\begin{table}[H]
    \centering
    \begin{tabular}{ll}
    \toprule
    ファイル名 & 役割 \\
    \midrule
    app5.js & メインロジック(ルーティング,CRUD処理) \\
    artifact.ejs & 聖遺物セット一覧表示画面テンプレート \\
    artifact\_set\_new.html & 聖遺物セット名称新規登録画面テンプレート \\
    artifact\_set.ejs & 聖遺物セット部位一覧画面テンプレート \\
    artifact\_item\_new.ejs & 聖遺物部位新規登録画面テンプレート \\
    artifact\_item\_detail.ejs & 聖遺物部位詳細表示画面テンプレート \\
    artifact\_item\_edit.ejs & 聖遺物部位編集画面テンプレート \\
    \bottomrule
    \end{tabular}
    \caption{聖遺物一覧表示システムのファイル構成}
\end{table}

\section{ページ遷移について}
各システムのページ遷移は以下の通りである.
\begin{figure}[H]
    \centering
    \includegraphics[clip,viewport=30 550 580 780,width=0.9\textwidth]{chara.pdf}
    \caption{星5キャラクター一覧表示のページ遷移図}
    \label{fig:page_transition}
\end{figure}
\begin{figure}[H]
    \centering
    \includegraphics[width=0.9\textwidth]{weapon.pdf}
    \caption{星4・5武器一覧表示のページ遷移図}
    \label{fig:page_transition_weapon}
\end{figure}
\begin{figure}[H]
    \centering
    \includegraphics[width=0.9\textwidth]{artifact.pdf}
    \caption{聖遺物一覧表示のページ遷移図}
    \label{fig:page_transition_artifact}
\end{figure}
ページ遷移について,各システム共通の仕様として,最初の一覧表示画面から各詳細画面へ遷移し,
詳細画面から編集画面へ遷移する形と,一覧表示から新規登録画面へ遷移する形を採用している.
星5キャラクター一覧表示および星4・5武器一覧表示では,まず一覧表示画面から詳細画面へ遷移し,
詳細画面から編集画面へ遷移する形を採用している.
一方,聖遺物一覧表示では,聖遺物セットごとに部位を管理しているため,
一覧表示画面からセットの部位一覧画面へ遷移し,部位一覧画面から詳細画面へ遷移する形を採用している.

\section{授業外から取り入れた技術}
新規データの登録時に,既存データと重複しない一意の \textsf{chara\_id} を発行するため,
次のアルゴリズムを実装した.
\begin{verbatim} 
    const id = character.length > 0 ? Math.max(...character.map(c => c.chara_id)) + 1 : 1; 
\end{verbatim}
このコードは,まず既存のキャラクターデータ配列 \texttt{character} の中から,
各要素の \textsf{chara\_id} を抽出し,最大値を求める.
その後,最大値に1を加えることで,新しいキャラクターのIDを生成する.
もし,既存データが存在しない場合は,初期値として1を割り当てる.
この方法により,常に一意のIDが生成され,データの整合性が保たれる.

また,URLパラメータ等で指定された管理番号に基づき,配列内から単一のオブジェクトを取得する処理には \textsf{find} メソッドを用いている.
例えば,星5キャラクターの詳細表示画面において,指定されたIDに対応するキャラクターデータを取得するコードは以下の通りである.
\begin{verbatim} 
    const detail = character.find(chara => chara.chara_id == id); 
\end{verbatim}
このコードは,配列 \texttt{character} の中から,条件に一致する最初の要素を返す.
ここでは,各キャラクターオブジェクトの \textsf{chara\_id} が指定されたIDと等しい場合に一致とみなしている.
この手法は,配列のインデックス番号に依存せず,論理的なIDに基づいてリソースを確実に特定できる利点がある.

削除処理については,指定された管理番号に基づき,配列内から該当オブジェクトを除外するために \textsf{filter} メソッドを使用している.
これは,元の配列を直接変更するのではなく,新しい配列を生成することで,指定されたIDに一致しない要素のみを保持する方法である.
\begin{verbatim} 
    character = character.filter(chara => chara.chara_id != id); 
\end{verbatim}
このコードは,配列 \texttt{character} の中から,条件に一致しない要素を抽出し,新しい配列として再代入している.
ここでは,各キャラクターオブジェクトの \textsf{chara\_id} が指定されたIDと等しくない場合にのみ,新しい配列に含める.
この方法により,指定されたIDのキャラクターが配列から削除される.

\end{document}